% Options for packages loaded elsewhere
\PassOptionsToPackage{unicode}{hyperref}
\PassOptionsToPackage{hyphens}{url}
%
\documentclass[
]{article}
\title{Tugas Simple Regression}
\author{Anak Agung Ayu Diva Shanty Darmawan}
\date{1/31/2022}

\usepackage{amsmath,amssymb}
\usepackage{lmodern}
\usepackage{iftex}
\ifPDFTeX
  \usepackage[T1]{fontenc}
  \usepackage[utf8]{inputenc}
  \usepackage{textcomp} % provide euro and other symbols
\else % if luatex or xetex
  \usepackage{unicode-math}
  \defaultfontfeatures{Scale=MatchLowercase}
  \defaultfontfeatures[\rmfamily]{Ligatures=TeX,Scale=1}
\fi
% Use upquote if available, for straight quotes in verbatim environments
\IfFileExists{upquote.sty}{\usepackage{upquote}}{}
\IfFileExists{microtype.sty}{% use microtype if available
  \usepackage[]{microtype}
  \UseMicrotypeSet[protrusion]{basicmath} % disable protrusion for tt fonts
}{}
\makeatletter
\@ifundefined{KOMAClassName}{% if non-KOMA class
  \IfFileExists{parskip.sty}{%
    \usepackage{parskip}
  }{% else
    \setlength{\parindent}{0pt}
    \setlength{\parskip}{6pt plus 2pt minus 1pt}}
}{% if KOMA class
  \KOMAoptions{parskip=half}}
\makeatother
\usepackage{xcolor}
\IfFileExists{xurl.sty}{\usepackage{xurl}}{} % add URL line breaks if available
\IfFileExists{bookmark.sty}{\usepackage{bookmark}}{\usepackage{hyperref}}
\hypersetup{
  pdftitle={Tugas Simple Regression},
  pdfauthor={Anak Agung Ayu Diva Shanty Darmawan},
  hidelinks,
  pdfcreator={LaTeX via pandoc}}
\urlstyle{same} % disable monospaced font for URLs
\usepackage[margin=1in]{geometry}
\usepackage{color}
\usepackage{fancyvrb}
\newcommand{\VerbBar}{|}
\newcommand{\VERB}{\Verb[commandchars=\\\{\}]}
\DefineVerbatimEnvironment{Highlighting}{Verbatim}{commandchars=\\\{\}}
% Add ',fontsize=\small' for more characters per line
\usepackage{framed}
\definecolor{shadecolor}{RGB}{248,248,248}
\newenvironment{Shaded}{\begin{snugshade}}{\end{snugshade}}
\newcommand{\AlertTok}[1]{\textcolor[rgb]{0.94,0.16,0.16}{#1}}
\newcommand{\AnnotationTok}[1]{\textcolor[rgb]{0.56,0.35,0.01}{\textbf{\textit{#1}}}}
\newcommand{\AttributeTok}[1]{\textcolor[rgb]{0.77,0.63,0.00}{#1}}
\newcommand{\BaseNTok}[1]{\textcolor[rgb]{0.00,0.00,0.81}{#1}}
\newcommand{\BuiltInTok}[1]{#1}
\newcommand{\CharTok}[1]{\textcolor[rgb]{0.31,0.60,0.02}{#1}}
\newcommand{\CommentTok}[1]{\textcolor[rgb]{0.56,0.35,0.01}{\textit{#1}}}
\newcommand{\CommentVarTok}[1]{\textcolor[rgb]{0.56,0.35,0.01}{\textbf{\textit{#1}}}}
\newcommand{\ConstantTok}[1]{\textcolor[rgb]{0.00,0.00,0.00}{#1}}
\newcommand{\ControlFlowTok}[1]{\textcolor[rgb]{0.13,0.29,0.53}{\textbf{#1}}}
\newcommand{\DataTypeTok}[1]{\textcolor[rgb]{0.13,0.29,0.53}{#1}}
\newcommand{\DecValTok}[1]{\textcolor[rgb]{0.00,0.00,0.81}{#1}}
\newcommand{\DocumentationTok}[1]{\textcolor[rgb]{0.56,0.35,0.01}{\textbf{\textit{#1}}}}
\newcommand{\ErrorTok}[1]{\textcolor[rgb]{0.64,0.00,0.00}{\textbf{#1}}}
\newcommand{\ExtensionTok}[1]{#1}
\newcommand{\FloatTok}[1]{\textcolor[rgb]{0.00,0.00,0.81}{#1}}
\newcommand{\FunctionTok}[1]{\textcolor[rgb]{0.00,0.00,0.00}{#1}}
\newcommand{\ImportTok}[1]{#1}
\newcommand{\InformationTok}[1]{\textcolor[rgb]{0.56,0.35,0.01}{\textbf{\textit{#1}}}}
\newcommand{\KeywordTok}[1]{\textcolor[rgb]{0.13,0.29,0.53}{\textbf{#1}}}
\newcommand{\NormalTok}[1]{#1}
\newcommand{\OperatorTok}[1]{\textcolor[rgb]{0.81,0.36,0.00}{\textbf{#1}}}
\newcommand{\OtherTok}[1]{\textcolor[rgb]{0.56,0.35,0.01}{#1}}
\newcommand{\PreprocessorTok}[1]{\textcolor[rgb]{0.56,0.35,0.01}{\textit{#1}}}
\newcommand{\RegionMarkerTok}[1]{#1}
\newcommand{\SpecialCharTok}[1]{\textcolor[rgb]{0.00,0.00,0.00}{#1}}
\newcommand{\SpecialStringTok}[1]{\textcolor[rgb]{0.31,0.60,0.02}{#1}}
\newcommand{\StringTok}[1]{\textcolor[rgb]{0.31,0.60,0.02}{#1}}
\newcommand{\VariableTok}[1]{\textcolor[rgb]{0.00,0.00,0.00}{#1}}
\newcommand{\VerbatimStringTok}[1]{\textcolor[rgb]{0.31,0.60,0.02}{#1}}
\newcommand{\WarningTok}[1]{\textcolor[rgb]{0.56,0.35,0.01}{\textbf{\textit{#1}}}}
\usepackage{graphicx}
\makeatletter
\def\maxwidth{\ifdim\Gin@nat@width>\linewidth\linewidth\else\Gin@nat@width\fi}
\def\maxheight{\ifdim\Gin@nat@height>\textheight\textheight\else\Gin@nat@height\fi}
\makeatother
% Scale images if necessary, so that they will not overflow the page
% margins by default, and it is still possible to overwrite the defaults
% using explicit options in \includegraphics[width, height, ...]{}
\setkeys{Gin}{width=\maxwidth,height=\maxheight,keepaspectratio}
% Set default figure placement to htbp
\makeatletter
\def\fps@figure{htbp}
\makeatother
\setlength{\emergencystretch}{3em} % prevent overfull lines
\providecommand{\tightlist}{%
  \setlength{\itemsep}{0pt}\setlength{\parskip}{0pt}}
\setcounter{secnumdepth}{-\maxdimen} % remove section numbering
\ifLuaTeX
  \usepackage{selnolig}  % disable illegal ligatures
\fi

\begin{document}
\maketitle

{
\setcounter{tocdepth}{2}
\tableofcontents
}
Berikut adalah data-data yang akan digunakan.

\begin{Shaded}
\begin{Highlighting}[]
\FunctionTok{library}\NormalTok{(ggplot2)}
\NormalTok{mfc }\OtherTok{=} \FunctionTok{read.csv}\NormalTok{(}\StringTok{\textquotesingle{}MFC.csv\textquotesingle{}}\NormalTok{, }\AttributeTok{stringsAsFactors =} \ConstantTok{FALSE}\NormalTok{)}
\NormalTok{sp500 }\OtherTok{=} \FunctionTok{read.csv}\NormalTok{(}\StringTok{\textquotesingle{}sp500.csv\textquotesingle{}}\NormalTok{, }\AttributeTok{stringsAsFactors =} \ConstantTok{FALSE}\NormalTok{)}
\NormalTok{irx }\OtherTok{=} \FunctionTok{read.csv}\NormalTok{(}\StringTok{\textquotesingle{}irx.csv\textquotesingle{}}\NormalTok{, }\AttributeTok{stringsAsFactors =} \ConstantTok{FALSE}\NormalTok{)}
\end{Highlighting}
\end{Shaded}

\hypertarget{jawaban-nomor-1}{%
\section{Jawaban nomor 1}\label{jawaban-nomor-1}}

Buatlah data frame baru yang berisi dua variabel, excess return
individual dari saham Manulife dan excess return market S\&P500 mulai
dari Januari 2005 sampai Desember 2014.

Untuk mendapatkan excess return dapat menggunakan model CPAM.

\begin{Shaded}
\begin{Highlighting}[]
\NormalTok{index }\OtherTok{=} \FunctionTok{seq}\NormalTok{(}\AttributeTok{from =} \DecValTok{127}\NormalTok{, }\AttributeTok{to=} \DecValTok{8}\NormalTok{, }\AttributeTok{by =} \SpecialCharTok{{-}}\DecValTok{1}\NormalTok{)}
\NormalTok{mfcreturn }\OtherTok{=}\NormalTok{ (mfc}\SpecialCharTok{$}\NormalTok{Adj.Close[index]}\SpecialCharTok{/}\NormalTok{mfc}\SpecialCharTok{$}\NormalTok{Adj.Close[index }\SpecialCharTok{+} \DecValTok{1}\NormalTok{])}\SpecialCharTok{{-}} \DecValTok{1} 
\NormalTok{spreturn }\OtherTok{=}\NormalTok{ (sp500}\SpecialCharTok{$}\NormalTok{Adj.Close[index]}\SpecialCharTok{/}\NormalTok{sp500}\SpecialCharTok{$}\NormalTok{Adj.Close[index }\SpecialCharTok{+} \DecValTok{1}\NormalTok{]) }\SpecialCharTok{{-}} \DecValTok{1}
\NormalTok{rf\_rate }\OtherTok{=}\NormalTok{ irx}\SpecialCharTok{$}\NormalTok{Adj.Close[index }\SpecialCharTok{+} \DecValTok{1}\NormalTok{]}\SpecialCharTok{/}\DecValTok{1200}
\NormalTok{ROR\_mfc }\OtherTok{=}\NormalTok{ mfcreturn }\SpecialCharTok{{-}}\NormalTok{ rf\_rate}
\NormalTok{ROR\_sp }\OtherTok{=}\NormalTok{ spreturn }\SpecialCharTok{{-}}\NormalTok{ rf\_rate}
\NormalTok{data\_10 }\OtherTok{=} \FunctionTok{data.frame}\NormalTok{(}\AttributeTok{individu =}\NormalTok{ ROR\_mfc, }\AttributeTok{market =}\NormalTok{ ROR\_sp)}
\FunctionTok{head}\NormalTok{(data\_10)}
\end{Highlighting}
\end{Shaded}

\begin{verbatim}
##       individu       market
## 1  0.062016836  0.016885013
## 2  0.030516898 -0.021359322
## 3 -0.045465272 -0.022376915
## 4  0.006420319  0.027587880
## 5  0.038077092 -0.002544381
## 6  0.049531119  0.033418287
\end{verbatim}

Buatlah scatterplot dari data tersebut.

\begin{Shaded}
\begin{Highlighting}[]
\FunctionTok{plot}\NormalTok{(data\_10}\SpecialCharTok{$}\NormalTok{market, data\_10}\SpecialCharTok{$}\NormalTok{individu, }\AttributeTok{col =} \StringTok{"blue"}\NormalTok{)}
\end{Highlighting}
\end{Shaded}

\includegraphics{PR-Simple-Regresi_files/figure-latex/unnamed-chunk-3-1.pdf}

\hypertarget{jawaban-nomor-2}{%
\section{Jawaban nomor 2}\label{jawaban-nomor-2}}

Buatlah data training dan data testing dari data pada nomor 1.

\begin{Shaded}
\begin{Highlighting}[]
\NormalTok{n.training }\OtherTok{=} \FunctionTok{seq}\NormalTok{(}\AttributeTok{from =} \DecValTok{128}\NormalTok{, }\AttributeTok{to =} \DecValTok{21}\NormalTok{, }\AttributeTok{by =} \SpecialCharTok{{-}}\DecValTok{1}\NormalTok{)}
\NormalTok{n.testing }\OtherTok{=} \FunctionTok{seq}\NormalTok{(}\AttributeTok{from =} \DecValTok{20}\NormalTok{, }\AttributeTok{to =} \DecValTok{9}\NormalTok{, }\AttributeTok{by =} \SpecialCharTok{{-}}\DecValTok{1}\NormalTok{)}

\NormalTok{training.key }\OtherTok{=} \FunctionTok{sample}\NormalTok{(}\FunctionTok{nrow}\NormalTok{(mfc), }\AttributeTok{size =}\NormalTok{ n.training, }\AttributeTok{replace =} \ConstantTok{FALSE}\NormalTok{, }\AttributeTok{prob =} \ConstantTok{NULL}\NormalTok{)}
\NormalTok{testing.key }\OtherTok{=} \FunctionTok{sample}\NormalTok{(}\FunctionTok{nrow}\NormalTok{(mfc), }\AttributeTok{size =}\NormalTok{ n.testing, }\AttributeTok{replace =} \ConstantTok{FALSE}\NormalTok{, }\AttributeTok{prob =} \ConstantTok{NULL}\NormalTok{)}

\NormalTok{training\_data }\OtherTok{=}\NormalTok{ mfc[training.key,]}
\NormalTok{testing\_data }\OtherTok{=}\NormalTok{ mfc[}\SpecialCharTok{{-}}\NormalTok{testing.key,]}
\end{Highlighting}
\end{Shaded}

Buatlah model regresi.

\begin{Shaded}
\begin{Highlighting}[]
\NormalTok{regresi1 }\OtherTok{=} \FunctionTok{lm}\NormalTok{(data\_10}\SpecialCharTok{$}\NormalTok{individu }\SpecialCharTok{\textasciitilde{}}\NormalTok{ data\_10}\SpecialCharTok{$}\NormalTok{market, }\AttributeTok{data =}\NormalTok{ data\_10)}
\FunctionTok{plot}\NormalTok{(data\_10}\SpecialCharTok{$}\NormalTok{market, data\_10}\SpecialCharTok{$}\NormalTok{individu, }\AttributeTok{col =} \StringTok{"blue"}\NormalTok{)}
\end{Highlighting}
\end{Shaded}

\includegraphics{PR-Simple-Regresi_files/figure-latex/unnamed-chunk-5-1.pdf}
Lakukan prediksi.

\begin{Shaded}
\begin{Highlighting}[]
\FunctionTok{summary}\NormalTok{(regresi1)}
\end{Highlighting}
\end{Shaded}

\begin{verbatim}
## 
## Call:
## lm(formula = data_10$individu ~ data_10$market, data = data_10)
## 
## Residuals:
##      Min       1Q   Median       3Q      Max 
## -0.21652 -0.04091 -0.00232  0.04084  0.34572 
## 
## Coefficients:
##                 Estimate Std. Error t value Pr(>|t|)    
## (Intercept)    -0.002436   0.007262  -0.335    0.738    
## data_10$market  1.955416   0.170676  11.457   <2e-16 ***
## ---
## Signif. codes:  0 '***' 0.001 '**' 0.01 '*' 0.05 '.' 0.1 ' ' 1
## 
## Residual standard error: 0.07917 on 118 degrees of freedom
## Multiple R-squared:  0.5266, Adjusted R-squared:  0.5226 
## F-statistic: 131.3 on 1 and 118 DF,  p-value: < 2.2e-16
\end{verbatim}

\begin{Shaded}
\begin{Highlighting}[]
\NormalTok{prediksi }\OtherTok{=} \FunctionTok{predict}\NormalTok{(regresi1, }\AttributeTok{newdata =}\NormalTok{ testing\_data)}
\end{Highlighting}
\end{Shaded}

\begin{verbatim}
## Warning: 'newdata' had 172 rows but variables found have 120 rows
\end{verbatim}

\begin{Shaded}
\begin{Highlighting}[]
\NormalTok{data\_10}\SpecialCharTok{$}\NormalTok{Predicted }\OtherTok{=}\NormalTok{ prediksi}
\FunctionTok{View}\NormalTok{(data\_10)}
\end{Highlighting}
\end{Shaded}

\hypertarget{jawaban-nomor-3}{%
\section{Jawaban nomor 3}\label{jawaban-nomor-3}}

Buatlah data frame baru dengan tiga variabe;: excess return Manulife,
excess return S\&P500, dan variabel kategorikal data yang diambil
sebelum 1 September 2008 (BEFORE) dan sesudah (AFTER).

\begin{Shaded}
\begin{Highlighting}[]
\NormalTok{index2 }\OtherTok{=} \FunctionTok{seq}\NormalTok{(}\AttributeTok{from =} \DecValTok{187}\NormalTok{, }\AttributeTok{to=} \DecValTok{2}\NormalTok{, }\AttributeTok{by =} \SpecialCharTok{{-}}\DecValTok{1}\NormalTok{)}
\NormalTok{mfcreturn\_2 }\OtherTok{=}\NormalTok{ (mfc}\SpecialCharTok{$}\NormalTok{Adj.Close[index2]}\SpecialCharTok{/}\NormalTok{mfc}\SpecialCharTok{$}\NormalTok{Adj.Close[index2 }\SpecialCharTok{+} \DecValTok{1}\NormalTok{])}\SpecialCharTok{{-}} \DecValTok{1} 
\NormalTok{spreturn\_2 }\OtherTok{=}\NormalTok{ (sp500}\SpecialCharTok{$}\NormalTok{Adj.Close[index2]}\SpecialCharTok{/}\NormalTok{sp500}\SpecialCharTok{$}\NormalTok{Adj.Close[index2 }\SpecialCharTok{+} \DecValTok{1}\NormalTok{]) }\SpecialCharTok{{-}} \DecValTok{1}
\NormalTok{rf\_rate2 }\OtherTok{=}\NormalTok{ irx}\SpecialCharTok{$}\NormalTok{Adj.Close[index2 }\SpecialCharTok{+} \DecValTok{1}\NormalTok{]}\SpecialCharTok{/}\DecValTok{1200}
\NormalTok{ROR\_mfc\_2 }\OtherTok{=}\NormalTok{ mfcreturn\_2 }\SpecialCharTok{{-}}\NormalTok{ rf\_rate2}
\NormalTok{ROR\_sp\_2 }\OtherTok{=}\NormalTok{ spreturn\_2 }\SpecialCharTok{{-}}\NormalTok{ rf\_rate2}
\NormalTok{tgl }\OtherTok{=}\NormalTok{ mfc}\SpecialCharTok{$}\NormalTok{Date[index2 }\SpecialCharTok{+} \DecValTok{1}\NormalTok{]}
\NormalTok{data2\_10 }\OtherTok{=} \FunctionTok{data.frame}\NormalTok{(tgl, ROR\_mfc\_2, ROR\_sp\_2) }

\NormalTok{tgl\_2 }\OtherTok{=}\NormalTok{ (}\StringTok{"2008{-}09{-}01"}\NormalTok{)}
\NormalTok{tgl\_baru }\OtherTok{=} \FunctionTok{as.Date}\NormalTok{(tgl\_2)}
\NormalTok{data2\_10}\SpecialCharTok{$}\NormalTok{tgl }\OtherTok{=} \FunctionTok{as.Date}\NormalTok{(data2\_10}\SpecialCharTok{$}\NormalTok{tgl)}

\ControlFlowTok{for}\NormalTok{(i }\ControlFlowTok{in} \DecValTok{1}\SpecialCharTok{:}\FunctionTok{nrow}\NormalTok{(data2\_10))\{}
  \ControlFlowTok{if}\NormalTok{(data2\_10}\SpecialCharTok{$}\NormalTok{tgl[i] }\SpecialCharTok{\textless{}}\NormalTok{ tgl\_baru)\{}
\NormalTok{    data2\_10}\SpecialCharTok{$}\NormalTok{status[i] }\OtherTok{=} \StringTok{"Before"}
\NormalTok{  \}}
  \ControlFlowTok{else}\NormalTok{\{}
\NormalTok{    data2\_10}\SpecialCharTok{$}\NormalTok{status[i] }\OtherTok{=} \StringTok{"After"}
\NormalTok{  \}}
\NormalTok{\}}

\FunctionTok{View}\NormalTok{(data2\_10)}
\end{Highlighting}
\end{Shaded}

Scatterplot dari data tersebut.

\begin{Shaded}
\begin{Highlighting}[]
\FunctionTok{ggplot}\NormalTok{(}\AttributeTok{data =}\NormalTok{ data2\_10) }\SpecialCharTok{+} \FunctionTok{geom\_point}\NormalTok{(}\AttributeTok{mapping =} \FunctionTok{aes}\NormalTok{(}\AttributeTok{x =}\NormalTok{ spreturn\_2, }\AttributeTok{y =}\NormalTok{ mfcreturn\_2, }\AttributeTok{color =}\NormalTok{ tgl))}
\end{Highlighting}
\end{Shaded}

\includegraphics{PR-Simple-Regresi_files/figure-latex/unnamed-chunk-8-1.pdf}

\hypertarget{jawaban-nomor-4}{%
\section{Jawaban nomor 4}\label{jawaban-nomor-4}}

Buatlah dua model CPAM dengan model yang menggunakan data sebelum 1
September 2008 dan menggunakan data sesudah 1 September 2008.

\hypertarget{sebelum-1-september-2008.}{%
\subsection{Sebelum 1 September 2008.}\label{sebelum-1-september-2008.}}

\begin{Shaded}
\begin{Highlighting}[]
\NormalTok{index3 }\OtherTok{=} \FunctionTok{seq}\NormalTok{(}\AttributeTok{from =} \DecValTok{104}\NormalTok{, }\AttributeTok{to=} \DecValTok{1}\NormalTok{, }\AttributeTok{by =} \SpecialCharTok{{-}}\DecValTok{1}\NormalTok{)}
\NormalTok{mfcreturn\_3 }\OtherTok{=}\NormalTok{ (mfc}\SpecialCharTok{$}\NormalTok{Adj.Close[index3]}\SpecialCharTok{/}\NormalTok{mfc}\SpecialCharTok{$}\NormalTok{Adj.Close[index3 }\SpecialCharTok{+} \DecValTok{1}\NormalTok{])}\SpecialCharTok{{-}} \DecValTok{1} 
\NormalTok{spreturn\_3 }\OtherTok{=}\NormalTok{ (sp500}\SpecialCharTok{$}\NormalTok{Adj.Close[index3]}\SpecialCharTok{/}\NormalTok{sp500}\SpecialCharTok{$}\NormalTok{Adj.Close[index3 }\SpecialCharTok{+} \DecValTok{1}\NormalTok{]) }\SpecialCharTok{{-}} \DecValTok{1}
\NormalTok{rf\_rate3 }\OtherTok{=}\NormalTok{ irx}\SpecialCharTok{$}\NormalTok{Adj.Close[index3 }\SpecialCharTok{+} \DecValTok{1}\NormalTok{]}\SpecialCharTok{/}\DecValTok{1200}
\NormalTok{ROR\_mfc\_3 }\OtherTok{=}\NormalTok{ mfcreturn\_3 }\SpecialCharTok{{-}}\NormalTok{ rf\_rate3}
\NormalTok{ROR\_sp\_3 }\OtherTok{=}\NormalTok{ spreturn\_3 }\SpecialCharTok{{-}}\NormalTok{ rf\_rate3}
\NormalTok{data3\_10 }\OtherTok{=} \FunctionTok{data.frame}\NormalTok{(}\AttributeTok{individu =}\NormalTok{ ROR\_mfc\_3, }\AttributeTok{market =}\NormalTok{ ROR\_sp\_3)}
\FunctionTok{head}\NormalTok{(data3\_10)}
\end{Highlighting}
\end{Shaded}

\begin{verbatim}
##       individu       market
## 1 -0.006438452  0.009988209
## 2 -0.003361005 -0.025992009
## 3  0.022051448  0.005817483
## 4  0.044106564  0.039211524
## 5  0.036506383  0.028611729
## 6 -0.004895672 -0.021641322
\end{verbatim}

\begin{Shaded}
\begin{Highlighting}[]
\NormalTok{regresi2 }\OtherTok{=} \FunctionTok{lm}\NormalTok{(data3\_10}\SpecialCharTok{$}\NormalTok{individu}\SpecialCharTok{\textasciitilde{}}\NormalTok{data3\_10}\SpecialCharTok{$}\NormalTok{market, }\AttributeTok{data =}\NormalTok{ data3\_10)}
\FunctionTok{plot}\NormalTok{(data3\_10}\SpecialCharTok{$}\NormalTok{market, data3\_10}\SpecialCharTok{$}\NormalTok{individu, }\AttributeTok{col =} \StringTok{"blue"}\NormalTok{)}
\FunctionTok{abline}\NormalTok{(regresi2)}
\end{Highlighting}
\end{Shaded}

\includegraphics{PR-Simple-Regresi_files/figure-latex/unnamed-chunk-9-1.pdf}
\#\# Sesudah 1 September 2008

\begin{Shaded}
\begin{Highlighting}[]
\NormalTok{index4 }\OtherTok{=} \FunctionTok{seq}\NormalTok{(}\AttributeTok{from =} \DecValTok{186}\NormalTok{, }\AttributeTok{to=} \DecValTok{105}\NormalTok{, }\AttributeTok{by =} \SpecialCharTok{{-}}\DecValTok{1}\NormalTok{)}
\NormalTok{mfcreturn\_4 }\OtherTok{=}\NormalTok{ (mfc}\SpecialCharTok{$}\NormalTok{Adj.Close[index4]}\SpecialCharTok{/}\NormalTok{mfc}\SpecialCharTok{$}\NormalTok{Adj.Close[index4 }\SpecialCharTok{+} \DecValTok{1}\NormalTok{])}\SpecialCharTok{{-}} \DecValTok{1} 
\NormalTok{spreturn\_4 }\OtherTok{=}\NormalTok{ (sp500}\SpecialCharTok{$}\NormalTok{Adj.Close[index4]}\SpecialCharTok{/}\NormalTok{sp500}\SpecialCharTok{$}\NormalTok{Adj.Close[index4 }\SpecialCharTok{+} \DecValTok{1}\NormalTok{]) }\SpecialCharTok{{-}} \DecValTok{1}
\NormalTok{rf\_rate4 }\OtherTok{=}\NormalTok{ irx}\SpecialCharTok{$}\NormalTok{Adj.Close[index4 }\SpecialCharTok{+} \DecValTok{1}\NormalTok{]}\SpecialCharTok{/}\DecValTok{1200}
\NormalTok{ROR\_mfc\_4 }\OtherTok{=}\NormalTok{ mfcreturn\_4 }\SpecialCharTok{{-}}\NormalTok{ rf\_rate4}
\NormalTok{ROR\_sp\_4 }\OtherTok{=}\NormalTok{ spreturn\_4 }\SpecialCharTok{{-}}\NormalTok{ rf\_rate4}
\NormalTok{data4\_10 }\OtherTok{=} \FunctionTok{data.frame}\NormalTok{(}\AttributeTok{individu =}\NormalTok{ ROR\_mfc\_4, }\AttributeTok{market =}\NormalTok{ ROR\_sp\_4)}
\FunctionTok{head}\NormalTok{(data4\_10)}
\end{Highlighting}
\end{Shaded}

\begin{verbatim}
##       individu      market
## 1  0.184583238  0.09201983
## 2  0.059063188 -0.03556242
## 3  0.130016075 -0.02662338
## 4  0.006063222  0.01935855
## 5 -0.008258805 -0.02109128
## 6  0.212617698  0.05568244
\end{verbatim}

\begin{Shaded}
\begin{Highlighting}[]
\NormalTok{regresi3 }\OtherTok{=} \FunctionTok{lm}\NormalTok{(data4\_10}\SpecialCharTok{$}\NormalTok{individu}\SpecialCharTok{\textasciitilde{}}\NormalTok{data4\_10}\SpecialCharTok{$}\NormalTok{market, }\AttributeTok{data =}\NormalTok{ data4\_10)}
\FunctionTok{plot}\NormalTok{(data4\_10}\SpecialCharTok{$}\NormalTok{market, data4\_10}\SpecialCharTok{$}\NormalTok{individu, }\AttributeTok{col =} \StringTok{"blue"}\NormalTok{)}
\FunctionTok{abline}\NormalTok{(regresi3)}
\end{Highlighting}
\end{Shaded}

\includegraphics{PR-Simple-Regresi_files/figure-latex/unnamed-chunk-10-1.pdf}

\hypertarget{jawaban-nomor-5}{%
\section{Jawaban nomor 5}\label{jawaban-nomor-5}}

Dari ketiga model, buatlah kesimpulan model manakah yang terbaik.

\begin{Shaded}
\begin{Highlighting}[]
\FunctionTok{summary}\NormalTok{(regresi1)}
\end{Highlighting}
\end{Shaded}

\begin{verbatim}
## 
## Call:
## lm(formula = data_10$individu ~ data_10$market, data = data_10)
## 
## Residuals:
##      Min       1Q   Median       3Q      Max 
## -0.21652 -0.04091 -0.00232  0.04084  0.34572 
## 
## Coefficients:
##                 Estimate Std. Error t value Pr(>|t|)    
## (Intercept)    -0.002436   0.007262  -0.335    0.738    
## data_10$market  1.955416   0.170676  11.457   <2e-16 ***
## ---
## Signif. codes:  0 '***' 0.001 '**' 0.01 '*' 0.05 '.' 0.1 ' ' 1
## 
## Residual standard error: 0.07917 on 118 degrees of freedom
## Multiple R-squared:  0.5266, Adjusted R-squared:  0.5226 
## F-statistic: 131.3 on 1 and 118 DF,  p-value: < 2.2e-16
\end{verbatim}

\begin{Shaded}
\begin{Highlighting}[]
\FunctionTok{summary}\NormalTok{(regresi2)}
\end{Highlighting}
\end{Shaded}

\begin{verbatim}
## 
## Call:
## lm(formula = data3_10$individu ~ data3_10$market, data = data3_10)
## 
## Residuals:
##      Min       1Q   Median       3Q      Max 
## -0.21502 -0.04155 -0.00134  0.04765  0.34681 
## 
## Coefficients:
##                  Estimate Std. Error t value Pr(>|t|)    
## (Intercept)     -0.004165   0.008257  -0.504    0.615    
## data3_10$market  1.962257   0.179842  10.911   <2e-16 ***
## ---
## Signif. codes:  0 '***' 0.001 '**' 0.01 '*' 0.05 '.' 0.1 ' ' 1
## 
## Residual standard error: 0.08397 on 102 degrees of freedom
## Multiple R-squared:  0.5386, Adjusted R-squared:  0.534 
## F-statistic:   119 on 1 and 102 DF,  p-value: < 2.2e-16
\end{verbatim}

\begin{Shaded}
\begin{Highlighting}[]
\FunctionTok{summary}\NormalTok{(regresi3)}
\end{Highlighting}
\end{Shaded}

\begin{verbatim}
## 
## Call:
## lm(formula = data4_10$individu ~ data4_10$market, data = data4_10)
## 
## Residuals:
##       Min        1Q    Median        3Q       Max 
## -0.169251 -0.034006 -0.001101  0.027900  0.220094 
## 
## Coefficients:
##                 Estimate Std. Error t value Pr(>|t|)    
## (Intercept)     0.023732   0.007104   3.341  0.00127 ** 
## data4_10$market 0.790732   0.172605   4.581 1.68e-05 ***
## ---
## Signif. codes:  0 '***' 0.001 '**' 0.01 '*' 0.05 '.' 0.1 ' ' 1
## 
## Residual standard error: 0.0643 on 80 degrees of freedom
## Multiple R-squared:  0.2078, Adjusted R-squared:  0.1979 
## F-statistic: 20.99 on 1 and 80 DF,  p-value: 1.678e-05
\end{verbatim}

Menurut saya model yang paling bagus adalah model regresi2 karena model
tersebut cukup baik dan dapat dikatakan normal dengan median yang hampir
0 serta mean dan max yang sesuai. Selain itu R squared nya juga lebih
tinggi dibandingkan 2 model lainnya yaitu 0.534.

\end{document}
